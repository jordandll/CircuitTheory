\documentclass{article}
\usepackage{amsmath}
\usepackage{amssymb}
\begin{document}
	Suppose we have a circuit with only one component, a capacitor with a 
	capacitance of $C$ farads, along with a power source supplying a voltage of 
	$E$ volts.
	$$ \frac{1}{C}\cdot\int i \cdot dt = E$$
	The charge, denoted as '$q$', as a function of time, is given by:
	$$ q(t) = \int i \cdot dt = C \cdot E(t)$$
	The current, denoted as '$i$', as a function of time is given by is given 
	by:
	$$ i(t) = C \cdot E'(t)$$
	\section[Resistance]{Negligible Resistance}
	The conductive materials that are used to construct circuits typically have 
	low, negligible, but still non-zero resistance.  While this can usually be 
	ignored with most applications, for demonstrative purposes, we will include 
	this resistance, denoted as '$R$', in a circuit equation.
	$$ Ri + \frac{1}{C} \int i \cdot dt = E$$
	The above is an integral-differential (sp?) equation.  To find the general 
	solution for $i$ we must differentiate throughout the equation with respect 
	to time.  First note that there are two scenarios that will be covered: one 
	where the voltage is constant; and one where it is variable.
	\subsection[Constant V.]{Constant Voltage}
	$$ Ri' + \frac{1}{C} i = 0$$
	Multiplying throughout by $C$ gives:
	$$ CRi' + i = 0$$
	The general solution is of the form:
	$$ i:=i(t)=Ae^{mt}$$
	Here, $A \in \mathbb{R}$ is an arbitrary constant to be solved for using an 
	IVP or IBP, and $m$ is the solution to the auxiliary equation (AE):
	$$ CRm + 1 = 0$$
	$$ m = -\frac{1}{CR}$$
	If the initial charge in the capacitor is zero then the IVP must be:
	$$ Ri(0) + 0 = E $$
	$$ RA = E$$
	$$ A = \frac{E}{R}$$
	As one can see by looking at the solution for $m$, the lower the 
	capacitance, the faster the current approaches zero.
	\section[Series]{Capacitors In Series}
	When there are two capacitors in series the total capacitance of the 
	circuit, denoted as '$C_T$', is given by:
	$$ C_T = \frac{1}{\frac{1}{C_1} + \frac{1}{C_2}}$$ 
	This makes more sense when looking at the following circuit equation.
	$$ Ri + \frac{1}{C_1} \int i \cdot dt + \frac{1}{C_2} \int i \cdot dt = E$$
	Via the axiom of distribution,
	$$ Ri + \left(\frac{1}{C_1} + \frac{1}{C_2}\right)\int i \cdot dt = E$$
	Via the above equation and the first equation of this section,
	$$ Ri + \frac{1}{C_T}\int i \cdot dt = E$$
	And, via the first equation of this section,
	$$ \frac{1}{C_T} = \frac{1}{C_1} + \frac{1}{C_2}$$
	Note that the above equations are extensible.  Assuming there are $n \in 
	\mathbb{N}$ capacitors in series, 
	$$ C_T = \frac{1}{\frac{1}{C_1} + \frac{1}{C_2} + \dots + 
	\frac{1}{C_n}}$$
	and 
	$$ \frac{1}{C_1} + \frac{1}{C_2} + \dots + \frac{1}{C_n}$$
	\section[Parallel]{Capacitors in Parallel}
	Suppose there are two capacitors, denoted as '$C1$' and '$C2$', each with a 
	capacitance of $C_1$ and $C_2$ respectively, in parallel with one another.  
	The total capacitance, denoted as $C_T$, of the parallel circuit is given 
	by:
	\begin{equation}\label{eq:parallel-capac}
		C_T = C_1 + C_2
	\end{equation}
	In a parallel circuit, the voltage running across each sub circuit is equal 
	to the voltage of the power supply.  In the case of this circuit, the 
	voltage running across $C1$ will be equal to the voltage running across 
	$C2$.  This is in stark contrast to a series circuit, in which case the 
	voltages of each capacitor would add up to the voltage of the power supply, 
	meaning they wouldn't be equal in voltage per se. \\ \\
	Assuming the voltage of the power supply of this parallel circuit is equal 
	to $E$ volts, the following holds true:
	\begin{equation}\label{eq:parallel-volts}
		\frac{1}{C_1}\int i_1 \cdot dt = \frac{1}{C_2} \int i_2 \cdot dt = E
	\end{equation}
	Suppose the paradigm is such that:
	\begin{itemize}
		\item $C1$ is the nearest to the power supply and is in between points 
		$1$ and $4$;
		\item $C2$ is in between points $2$ and $3$;
		\item And point $1$ is adjacent to the positive pole of the power 
		supply.
	\end{itemize}
	The current, denoted as '$I_T$', in between point $1$ and the positive pole 
	of the power supply, henceforth '$(+)$', is equal to the current in between 
	point $4$ and the negative pole of the power supply, '$(-)$'.  Both 
	quantities are given by:
	\begin{equation}\label{eq:parallel-cur}
		I_T = i_1 + i_2
	\end{equation}
	Differentiating throughout equation (\ref{eq:parallel-volts}) with respect 
	to time gives:
	\begin{equation}\tag{\ref{eq:parallel-volts}}
		\frac{1}{C_1} \cdot i_1 = \frac{1}{C_2} \cdot i_2 = E'(t)
	\end{equation}
	Via the above equation, equation (\ref{eq:parallel-cur}), and substitution,
	\begin{equation}\tag{\ref{eq:parallel-cur}}
		I_T = (C_1  + C_2) \cdot E'(t)
	\end{equation}
	And, via the above equation and equation (\ref{eq:parallel-capac}),
	\begin{equation}\tag{\ref{eq:parallel-cur}}
		I_T = C_T \cdot E'(t)
	\end{equation}
	Now consider the following:
	$$ E = \frac{1}{C_T} \int I_T \cdot dt$$
	The above fits the form of the standard equation that relates voltage with 
	capacitance and charge, which is equal to the primitive integral or 
	anti-derivative of the current. \\ \\
	If we divide throughout equation (\ref{eq:parallel-cur}) by $C_T$ and then 
	integrate throughout with respect to time we get the same equation that 
	relate capacitance and charge with voltage.
\end{document}
