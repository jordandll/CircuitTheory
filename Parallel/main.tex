\documentclass{article}
\usepackage{amsmath}
\usepackage{amssymb}
\begin{document}
	\section[RC-Circuit]{Resistor-Capacitor Circuits}
	Suppose there is a circuit with a battery supplying $E$ volts of energy and 
	a capacitor with a capacitance of $C$ farads in parallel with a resistor 
	with a resistance of $R$ ohms.  By applying Ohm's Law to this parallel 
	circuit, we get:
	$$ E=I_TR_T =(I_1+I_2)R$$
	Here, the total current, denoted as '$I_T$', is equal to the sum of the two 
	branch currents flowing through the capacitor and the resistor.  Said 
	branch currents are denoted as '$I_1$' and '$I_2$' respectively. \\ \\
	The four unknown quantities here are the aforementioned branch currents and 
	the voltages running across their respective circuit components.  To find 
	the 	values or the functions that give the values of these quantities -- 
	in other words, to solve for these quantities -- we will need four unique 
	equations.  Thankfully we have them.
	$$ V_C = \frac{1}{C}\int I_1 \cdot dt$$
	$$ V_R = RI_2$$
	$$ I_T = I_1 + I_2$$
	$$ I_T = \frac{E}{R}$$
	Via a basic principle of parallel circuits and the presupposition that the 
	resistor is in parallel with the capacitor,
	$$ V_C = V_R = RI_2 = \frac{1}{C} \int I_1 \cdot dt$$
	$$ CRI_2 = \int I_1 \cdot dt$$
	$$ I_1:=I_1(t) = CRI_2'(t)$$
	Via the above equation, the first equation of this section, and 
	substitution,
	$$ R(CRI_2'(t) + I_2(t) = E(t) := E$$
	The above is a differential equation with $I_2$ as the dependent variable 
	and time as the independent variable.  Assuming $E(t)$ is constant, the 
	general solution for $I_2(t)$ is:
	$$ I_2(t) = Ae^{mt} + \frac{E}{R}$$
	Here, $m$ is the solution to the auxiliary equation:
	$$ CRm + 1 = 0$$
	$$ m = - \frac{1}{CR}$$
	and $A \in \mathbb{R}$ is an arbitrary constant that can be solved for 
	using an IVP that involves the initial charge in the capacitor, denoted as 
	'$q_0$'.
	$$ q(t) = \int I_1 \cdot dt = CRI_2(t) = CRAe^{mt} + CE$$
	$$ q(0) = C(RA + E) = q_0$$
	$$ RA = \frac{q_0}{C} - E$$
	$$ A = \frac{1}{R}\left(\frac{q_0}{C}-E\right)$$
	This leaves us with a parametric specific solution for $I_2(t)$.
	$$ I_2 := I_2(t) := I_2(t \vert C, R, q_0) = Ae^{mt} + \frac{E}{R}$$
	If the initial charge is equal to zero Coloumbs then the current flowing 
	through the resistor will start off at zero amps and then rapidly increase 
	before plateauing at around what the measure of the current would be if 
	there were no capacitor to siphon some of the current.
	$$ i(t) := I_2(t \vert C, R, 0) = \frac{E}{R}(1-e^{mt})$$
	$$ \lim_{t \to \infty}i(t) = \frac{E}{R}$$
	\subsection{Applications}
	One common application of capacitors in parallel with other nodes of a 
	circuit is to provide stability to the circuit.  This quality is the result 
	of the 
	capacitor drawing or 'sucking' some of the excess current away from the 
	rest of the circuit during a spike in voltage.  The following equation 
	describes how that works and can be used to calculate the amount of current 
	that is drawn by the capacitor.
	$$ V_C = \frac{1}{C} \int I_C \cdot dt$$
	$$ C \cdot V_C = \int I_C \cdot dt$$
	$$ I_C = C \cdot V_C'(t) = C \cdot \frac{dV_C}{dt}$$
	As one can see in the above equation, the current is proportional to the 
	change in voltage running across the capacitor over time.  The faster the 
	voltage changes, the higher the magnitude of the current drawn by the 
	capacitor will be.
\end{document}
