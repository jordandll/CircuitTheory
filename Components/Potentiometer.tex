\documentclass{article}
\usepackage{amsmath}
\usepackage{amssymb}
\begin{document}
	In this document we examine the potentiometer in a number of circuits, 
	including in series, parallel, and singleton circuits (i.e. one with just 
	the potentiometer and a power supply).
	\section{Definition}
	A potentiometer is a passive device with three terminals and a variable 
	resistance of '$R_P$' ohms.  Said resistance is a function of the angle, 
	denoted as '$\theta$', of the knob attached to the device.
	\section{Presuppositions}
	In all of the circuit applications contained in this document there is 
	supposed to be a power supply with a voltage of $E$ volts.  Said voltage 
	may be variable and a function of time or constant.
	\section{Singleton}
	The circuit equation is:
	\begin{equation}\label{eq:V_P}
		E = V_P = R_P \cdot i
	\end{equation}
	Since the supplied voltage does not change in response to the angle of the 
	knob -- i.e. $E$ is not a function of $\theta$ -- only the current, denoted 
	as '$i$', can change as a function of $\theta$.
	$$ i(\theta) = \frac{E}{R_P(\theta)}$$
	If $E$ changes with time, we need to introduce time, denoted as '$t$', to 
	the equation and subsequently, the function that gives the current, making 
	it a multivariate function.
	$$ i(t, \theta) = \frac{E(t)}{R_P(\theta)}$$
	\section[Series]{In Series}
	As is the case with a singleton circuit, only the current changes with the 
	angle of the knob;  The supplied voltage is not a function of $\theta$.
	\subsection[Resistor]{With Another Resistor}
	Suppose there is a circuit with a pentiometer and a resistor with 
	a resistance of $R$ ohms.  The voltage running across this resistor is 
	proportional to the current running through the circuit.  This means that 
	changes in $\theta$ lead to changes in the voltage running across said 
	resistor and the potentiometer.  In other words, turning the knob 	
	\emph{divides} the voltage between the two components.
	\subsection[Capacitor]{With a Capacitor}
	Now suppose there is a circuit with a potentiometer and a capacitor with a 
	capacitance of $C$ farads.
	$$ E = V_P + V_C$$
	Here, '$V_C$' denotes the voltage running across the capacitor which is 
	given by:
	\begin{equation}\label{eq:V_C}
		V_C = \frac{1}{C} \int i \cdot dt
	\end{equation}
	Via the first equation of this subsection, the above equation, equation 
	(\ref{eq:V_P}), and substitution we end up with the following 
	integral-differential equation:
	$$ E = R_P \cdot i + \frac{1}{C} \int i \cdot dt$$
	To turn this into a differential equation that can be used to solve for the 
	current in terms of time we first must consider the function that gives $i$ 
	a uni variate function with $t$ as the variable and $\theta$ as the 
	parameter.  This is because the charge in the capacitor, which is a factor 
	of $V_C$, is equal to the primitive integral of the current with respect to 
	time;  Thus, to turn the above equation into a differential equation we 
	must \emph{fully} differentiate throughout as opposed to \emph{partially} 
	differentiate throughout with respect to time.  And doing so gives, 
	assuming the supplied voltage is constant, gives:
	$$ R_P \cdot i' + \frac{1}{C} i = 0$$
	The general solution is of the form:
	$$ i := i(t) := i(t \vert \theta) = Ae^{mt}$$
	where $A \in \mathbb{R}$ is an arbitrary constant to be solved for using 
	some initial value problem (IVP) and $m$ is the root of the auxiliary 
	equation:
	$$ R_P \cdot m + \frac{1}{C} = 0$$
	$$ CR_P \cdot m + 1 = 0$$
	$$ m := m(\theta) = - \frac{1}{CR_P(\theta)}$$
	Remember that $\theta$ is a parameter of the function that gives the 
	current.  In other words, $\theta$ is considered a constant when 
	differentiating or integrating with respect to the independent variable, 
	time or $t$. \\ \\
	For the IVP let $q$ be defined as:
	$$ q:=q(t) := \int i \cdot dt$$
	And, the IVP is:
	$$ E = R_P \cdot i(0) + \frac{1}{C} q(0)$$
	$$ E - \frac{1}{C}q(0) = R_P \cdot i(0) $$
	$$ i(0) = i(0 \vert \theta) = A(\theta) = \frac{E}{R_P(\theta)} + 
	m(\theta)q(0)$$
	\section[Parallel]{In Parallel}
	Recall that a potentiometer has three terminals.  These terminals usually 
	form a linear array.  Whatever shape they form, let the paradigm be 
	such that the first and last terminals are the poles of the voltage drop 
	when the potentiometer is in a series circuit.  The first and last 
	terminals are denoted as '$T$' and '$B$' respectively, while the middle 
	terminal is denoted as '$M$'. \\ \\
	With this paradigm, the magnitude of the voltage running across the 
	potentiometer is equal to the magnitude of the voltage drop from terminal 
	$T$ to terminal $B$.
	\begin{equation}\tag{\ref{eq:V_P}}
		|V_P| = |V_{TB}| = |R_P \cdot i|
	\end{equation}
	The above principle makes no assumption about the polarity of said voltage 
	drop.  Assuming $T$ has a polarity of $(+)$ and that this also the 
	polarity in the previous sections, the voltage drop from $T$ to $B$ is 
	equal to the voltage running across the potentiometer.
	\begin{equation}\tag{\ref{eq:V_P}}
		V_P = V_{TB} = R_P \cdot i
	\end{equation}
	\subsection[Resistor]{With Another Resistor}
	Suppose there is a parallel circuit with a potentiometer that is partially 
	in parallel and partially in series with a resistor with a resistance of 
	$R$ ohms.  We say 'partially' because $T$ to $M$ is in series with $M$ to 
	$B$ and with said resistor, while $M$ to $B$ is in parallel with the 
	resistor.
	$$ E = V_{TM} + V_R = V_{TM} + V_{MB} = V_P$$
	Breaking it down by each voltage drop, we have the following:
	$$ V_{TM} = R_{TM}(i_1 + i_2)$$
	$$ V_{MB} = R_{MB}\cdot i_1$$
	$$ V_R = R \cdot i_2$$
	Since there are two current to solve for, we need two unique equations.  
	Thankfully, via the first equation of this subsection, we have them.
	$$ V_{TM} + V_R = V_{TM} + V_{MB}$$
	$$ V_R = V_{MB}$$
	\begin{equation}\label{eq:branch-cur-1}
		R \cdot i_2 = R_{MB} \cdot i_1
	\end{equation}
	As for the second equation:
	$$ V_{TM} + V_{MB} = E$$
	\begin{equation}\label{eq:branch-cur-2}
		R_{TM}(i_1 + i_2) + R_{MB}\cdot i_1 = E
	\end{equation}
	\subsubsection[ResistP]{Find the Resistance of the Potentiometer}
	If both currents are already given and we simply want to find the 
	resistance of the potentiometer ($R_P$) then we can use the following 
	equation:
	$$ E = \frac{i_1 + i_2}{\frac{1}{R_P} + \frac{1}{R}}$$
	$$ E \left(\frac{1}{R_P} + \frac{1}{R} \right) = i_1 + i_2$$
	$$\frac{E}{R_P} = i_1 + i_2 - \frac{E}{R} $$
	$$ R_P = \frac{E}{i_1 + i_2 - \frac{E}{R}}$$
	\subsubsection[Currents]{Find the Branch Currents}
	To find the branch currents, allowing us to also find the total 
	resistance of the circuit, denoted as '$R_T$', we can use equations 
	(\ref{eq:branch-cur-1}) and (\ref{eq:branch-cur-2}). \\ \\
	Via equations (\ref{eq:branch-cur-1}), (\ref{eq:branch-cur-2}), and 
	substitution,
	$$ R_{TM}\cdot i_1\left(1 + \frac{R_{MB}}{R}\right) + R_{MB}\cdot i_1=E$$
	$$ R_{TM} \cdot i_1 \frac{R_{MB} + R}{R} + R_{MB} \cdot i_1 = E$$
	$$ \frac{E}{i_1} = \frac{R_{TM}(R_{MB} + R) + R_{MB}\cdot R}{R} = 
	\frac{(R_P-R_{MB})(R_{MB} + R) + R_{MB}\cdot R}{R}$$
	$$ \frac{E}{i_1} = \frac{R_PR_{MB} + R_PR - R_{MB}^2}{R}$$
	\begin{equation}\label{eq:branch-cur_1}
		i_1 = \frac{ER}{R_PR_{MB} + R_PR - R_{MB}^2}
	\end{equation}
	And via the above equation, equation (\ref{eq:branch-cur-1}), and 
	substitution, we can solve for $i_2$.
	\begin{equation}\label{eq:branch-cur_2}
		i_2 = \frac{ER_{MB}}{R_PR_{MB} + R_PR - R_{MB}^2}
	\end{equation}
	Finally, to find $R_T$, via Ohm's Law, as it applies to parallel circuits,
	$$ E = R_T(i_1 + i_2)$$
	Via the above equation divided throughout by $E$, equations 
	(\ref{eq:branch-cur_1}), (\ref{eq:branch-cur_2}), and substitution,
	$$ 1 = R_T \frac{R + R_{MB}}{R_PR_{MB} + R_PR - R_{MB}^2}$$
	\subsubsection[Mesh]{The Mesh Current Method}
	Herein we will use the mesh current method to find $i_1$, $i_2$, and 
	$R_T$.  Let '$I_1$' and '$I_2$' be defined as two mesh currents flowing in 
	the clockwise direction.  The first ($I_1$) one flows through the battery 
	($E$), $R_{TM}$, and $R$, while the second ($I_2$) flows through $R$ and 
	$R_{MB}$. \\ \\
	Via KVL,
	\begin{equation}\label{eq:mesh-cur-1}
		E-R_{TM}I_1-R(I_1-I_2) = 0
	\end{equation}
	\begin{equation}\label{eq:mesh-cur-2}
		R(I_2-I_1)+R_{MB}I_2 = 0
	\end{equation}
	\section[Resistance]{Finding the Resistance}
	Assuming the relationship between the angle, denoted as '$\theta$', of the knob and the resistance, denoted as '$R_P$', is linear, the following parametric function generalizes any specific (ie non-parametric) function that gives the resistance as a function of the angle.
	$$ R_P(\theta \vert \alpha, \beta) = \alpha \theta +\beta$$
	If the resistance is proportional to the angle then the parametric function that generalizes functions that are specific to such a hypothetical is given by:
	$$ R_P(\theta \vert \alpha) = R_P(\theta \vert \alpha, 0) = \alpha \theta$$
	Under these conditions, only one reliable test value needs to be passed as an argument to the function to find the leading coefficient of the specific function. \\ \\
	When the relationship is linear but not proportional, two reliable test values are needed.
	\subsection{Examples}
	Herein are some examples of finding the function that gives $R_P$ involving actual potentiometers and actual test results.  The instrument used to measure the resistance -- and any other electrical property -- is an Extech EX330 multimeter.
	\paragraph[Example 1]{The first example} involves a breadboard trim potentiometer - 10K.  Assuming a linear taper -- it isn't specified on the product page, so we will have to find out through experimentation -- the specific function that gives $R_P$ is:
	$$ R_P(\theta) := R_P(\theta \vert a, b) = a \theta + b$$
	The arguments passed to $R_P(\theta)$ during the first round of tests are $0$ and $0.25\pi$ radians. \\ \\
	The resistance from terminal $T$ to terminal $B$ is always equal to $R_P$, both of which were approximately equal to a constant $1 \text{ k} \Omega$ throughout testing.
	$$ R_P(0) \approx R_P(0.25\pi) \approx 1000 \Omega$$
	$$ R_{TB}(\theta) = R_P(\theta) \approx 1000 \Omega$$
	The measured resistance from terminals $T$ to $M$ and $M$ to $B$ unsurprisingly always added up to approximately the measured resistance of the potentiometer.  The aforementioned resistances did vary with the angle.
	\begin{equation}\label{eq:sum-resists}
		R_{TM} + R_{MB} \approx R_P
	\end{equation}
	$$ R_{MB}(0) \approx 511 \Omega$$
	$$ R_{MB}(0.25\pi) \approx 345 \Omega$$
	The above approximations can be used to approximate the coefficients of the functions that give $R_{TM}$ and $R_{MB}$.  First, let $R_{MB}(\theta)$ be defined as:
	$$ R_{MB}(\theta) := R_P(\theta \vert a, b) = a\theta + b$$ 
	Via the two pairs of measured values,
	$$ a \cdot 0 + b \approx 511 \Omega$$
	$$ b \approx 511 \Omega$$
	$$ a \cdot 0.25\pi + 511 \Omega \approx 345 \Omega$$
	$$ a \approx - \frac{664}{\pi} \Omega \approx -211.36 \Omega$$
	Via the above equations,
	$$ R_{MB}(\theta) \approx -211.36 \theta + 511$$
	Via the above equation and equation (\ref{eq:sum-resists}),
	$$ R_{TM}(\theta) \approx R_P(\theta) - R_{MB}(\theta) \approx 211.36 \theta + 489 $$
	Note that it is rational that the sum of the actual, as opposed to the measured, resistances is equal to the actual resistance of the potentiometer. 
\end{document}
