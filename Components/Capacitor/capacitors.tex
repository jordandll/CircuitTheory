\hypertarget{b11b46ae}{}
\begin{Shaded}
\begin{Highlighting}[]
\ImportTok{import}\NormalTok{ discharge}
\end{Highlighting}
\end{Shaded}

\hypertarget{e988cc97}{}
\hypertarget{capacitors}{%
\section{Capacitors}\label{capacitors}}

Suppose we have a circuit with only one component, a capacitor with a
capacitance of \(C\) farads, along with a power source supplying a
voltage of \(E\) volts. \[ \frac{1}{C} \int i \cdot dt = V_C = E\]

\hypertarget{in-parallel}{%
\subsection{In Parallel}\label{in-parallel}}

It then follows that the current drawn by the capacitor that is in
parallel with other nodes of the circuit is proportional to the
instantaneous rate of change of the voltage.
\[ i = C \frac{dV_C}{dt} = C V_C'(t)\] This property makes it a good
stabilizer. During a voltage spike, it will draw a current that is
proportional to said spike; Thus subtracting from the total current when
the voltage is increasing and adding to the total current when the
voltage is decreasing.

\hypertarget{capacitor-as-a-battery}{%
\subsection{Capacitor as a Battery}\label{capacitor-as-a-battery}}

Using a capacitor as the battery or power supply of a circuit is one way
of discharging one. Suppose there is a circuit consisting of only a
capacitor, serving as the battery or power supply of the circuit, with a
capacitance of \(C\) farads and an initial charge of \(q_0\) Coloumbs
and a resistor with a resistance of \(R\) ohms. The following equations
hold true: \[ V_C + V_R = \frac{1}{C}q(t) + Rq'(t) = 0\] The above is a
differential equation with the charge in the capacitor, denoted as
'\(q\)', as the dependent variable and time (\(t\)) as the independent
variable. As such, the general solution for \(q(t)\) is of the form:
\[ q(t) := q(t \vert C, R, q_0) = Ae^{mt}\] To solve for \(m\), we use
the auxiliary equation: \[Rm + \frac{1}{C} = 0\] \[CRm + 1 = 0\]
\[m = - \frac{1}{CR}\] And to solve for the arbitrary constant,
\(A \in \mathbb{R}\), we use the IVP:
\[ q(0) = Ae^{m \cdot 0} = A = q_0\] This leaves us with the specific
solution for the parametric function that gives the charge in the
capacitor as a function of time.
\textbackslash begin\{equation\}\textbackslash label\{eq:charge\} q(t
\textbackslash vert C, R, q\_0) = q\_0e\^{}\{mt\}
\textbackslash end\{equation\}

\hypertarget{c14a4ffd}{}
\begin{Shaded}
\begin{Highlighting}[]
\BuiltInTok{help}\NormalTok{(discharge.gen\_charge)}
\end{Highlighting}
\end{Shaded}

\begin{verbatim}
Help on function gen_charge in module discharge:

gen_charge(C, R, q_0)
    A parametric function that gives the charge in capacitor as a function
    of time.  The parameters are:
      C:    The capacitance in farads;
      R:    The resistance in ohms;
      q_0:  The initial charge in Coloumbs.

\end{verbatim}

\hypertarget{a058ce86}{}
\hypertarget{example}{%
\paragraph{Example}\label{example}}

This is a real life example. I have a \(50\)V \(10\mu\text{F}=10^{-5}\)F
capacitor that is fully charged. Via the aforementioned fact, the first
equation of this chapter, and substitution, the initial charge is:
\[ q_0 = 50 \text{V} \cdot 10^{-5}\text{F}=5 \cdot 10^{-4}\text{C}\]
Note that the 'C' in the above equation stands for Coloumbs, and is not
to be confused with '\(C\)', the denotation for the capacitance.\\
Let '\(q(t)'\) be defined as:
\[ q(t):=q(t \vert 10\mu \text{F}, 560\Omega, 5 \cdot 10^{-4}\text{C})\]
If this were used as a power supply to a circuit containing a resistor
with a resistance of \(560\Omega\), then the charge of a function of
time would be given by \(q(t)\)

\hypertarget{f72d140f}{}
\begin{Shaded}
\begin{Highlighting}[]
\NormalTok{q }\OperatorTok{=}\NormalTok{ discharge.gen\_charge(}\FloatTok{1e{-}5}\NormalTok{, }\DecValTok{560}\NormalTok{, }\FloatTok{5e{-}4}\NormalTok{)}
\NormalTok{q(}\FloatTok{0.0012}\NormalTok{)}
\end{Highlighting}
\end{Shaded}

\begin{verbatim}
0.00040355887350269464
\end{verbatim}

\leavevmode\vadjust pre{\hypertarget{903bde34}{}}%
As one can see with the code above, the charge goes down very quickly.
After just one micro second, the charge goes down by one order of
magnitude.

\hypertarget{74199df5}{}
\hypertarget{time-to-discharge}{%
\subsubsection{Time to Discharge}\label{time-to-discharge}}

The time it takes to discharge to \(X\)\% of the initial charge is given
by: Let \(t:=t(X):=t(X \vert R, C)\) \[ q_0 \cdot 0.01X = q_0e^{mt} \]
\[ 10^{-2}X = e^{mt}\] \[ \ln(X)-2\ln(10) = mt\]
\[ t = t(X) = - CR\big(\ln(X)-2\ln(10)\big) = CR(2\ln10 - \ln X)\]

\hypertarget{example}{%
\paragraph{Example}\label{example}}

This example uses the same capacitor as the previous example. Below is
how long it takes in milli seconds to discharge to 5\% of the initial
charge.

\hypertarget{4f58b735}{}
\begin{Shaded}
\begin{Highlighting}[]
\BuiltInTok{help}\NormalTok{(discharge.gen\_time)}
\end{Highlighting}
\end{Shaded}

\begin{verbatim}
Help on function gen_time in module discharge:

gen_time(C, R)
    A parametric function that gives the time it takes for the charge in a 
    capacitor with a capacitance of C farads to reach X% of it's initial charge.
    The returned function accepts one argument, the desired percent of the initial charge,
    denoted as 'X'.  Said function returns the time, in seconds, that it takes to reach
    X% of the initial charge.
    Parameter:
      C:    Capacitance in farads;;
      R:    Resistance in ohms.

\end{verbatim}

\hypertarget{5b365c38}{}
\begin{Shaded}
\begin{Highlighting}[]
\NormalTok{t }\OperatorTok{=}\NormalTok{ discharge.gen\_time(}\FloatTok{1e{-}5}\NormalTok{, }\DecValTok{560}\NormalTok{)}
\BuiltInTok{print}\NormalTok{(}\SpecialStringTok{f\textquotesingle{}t(10\%)=}\SpecialCharTok{\{}\NormalTok{t(}\DecValTok{10}\NormalTok{)}\SpecialCharTok{:.6f\}}\SpecialStringTok{s=}\SpecialCharTok{\{}\NormalTok{t(}\DecValTok{10}\NormalTok{)}\OperatorTok{*}\DecValTok{1000}\SpecialCharTok{:.3f\}}\SpecialStringTok{ms\textquotesingle{}}\NormalTok{)}
\end{Highlighting}
\end{Shaded}

\begin{verbatim}
t(10%)=0.012894s=12.894ms
\end{verbatim}

\leavevmode\vadjust pre{\hypertarget{6c7feb36}{}}%
Below is a curve given by \(q(t)\).

\hypertarget{8f63ead2}{}
\begin{Shaded}
\begin{Highlighting}[]
\ImportTok{import}\NormalTok{ numpy }\ImportTok{as}\NormalTok{ np}
\ImportTok{import}\NormalTok{ matplotlib.pyplot }\ImportTok{as}\NormalTok{ plt}
\NormalTok{X}\OperatorTok{=}\NormalTok{T}\OperatorTok{=}\NormalTok{np.linspace(}\FloatTok{0.0}\NormalTok{, }\FloatTok{0.013}\NormalTok{, num}\OperatorTok{=}\DecValTok{200}\NormalTok{)}
\NormalTok{Y}\OperatorTok{=}\NormalTok{Q}\OperatorTok{=}\NormalTok{np.array(}\BuiltInTok{list}\NormalTok{(}\BuiltInTok{map}\NormalTok{(q, T)))}
\NormalTok{mX}\OperatorTok{=}\NormalTok{X}\OperatorTok{*}\DecValTok{1000}
\NormalTok{mY}\OperatorTok{=}\NormalTok{Y}\OperatorTok{*}\DecValTok{1000}
\NormalTok{fig, ax }\OperatorTok{=}\NormalTok{ plt.subplots()}
\NormalTok{ax.set\_title(}\StringTok{\textquotesingle{}Charge in Capacitor\textquotesingle{}}\NormalTok{)}
\NormalTok{ax.set\_xlabel(}\StringTok{\textquotesingle{}time (ms)\textquotesingle{}}\NormalTok{)}
\NormalTok{ax.set\_ylabel(}\StringTok{\textquotesingle{}charge (milli Coloumbs)\textquotesingle{}}\NormalTok{)}
\NormalTok{ax.plot(mX, mY, label}\OperatorTok{=}\VerbatimStringTok{r\textquotesingle{}R=560 $\textbackslash{}Omega$, C=10$\textbackslash{}mu$F\textquotesingle{}}\NormalTok{)}
\NormalTok{ax.legend()}
\end{Highlighting}
\end{Shaded}

\begin{verbatim}
<matplotlib.legend.Legend at 0x7fe7a30c7be0>
\end{verbatim}

\includegraphics{6f613f9fc38f64643b75777765fb910f4504d5ae.png}
