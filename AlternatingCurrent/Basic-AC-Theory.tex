\documentclass{article}
\usepackage{amsmath}
\usepackage{amssymb}
\usepackage{mathtools}
\begin{document}
	\section[Amplitude]{Measurements of AC Magnitude}
	\subsection[Peak]{Peak Method}
	One way to express the magnitude or amplitude of an AC waveform is the peak 
	or peak to base paradigm.  Under this paradigm, the magnitude is equal to 
	the difference between the peak or crest value and the base value (which is 
	usually zero unless 
	the curve has been shifted along the y-axis). \\ \\
	Suppose there is a sine wave given by the function:
	$$ f(x):=f(x\vert A, \omega):=A\cdot \sin \omega 2\pi x\text{ , where }A, 
	\omega \ne 0$$
	The parameter $A \in \mathbb{R}\setminus\{0\}$ is equal to the magnitude of 
	the wave given by $f(x)$.  The other parameter, $\omega \in 
	\mathbb{R}\setminus \{0\}$, is equal to the frequency of the wave or curve 
	in Hertz.
	\subsection[PeakToPeak]{Peak to Peak Method}
	Another way of expressing the magnitude is known as the peak to peak (P-P) 
	paradigm.  Here, the magnitude is equal to the difference between one of 
	the crest values and the value of one of the troughs adjacent to said 
	crest. \\ \\
	Here, the parameter, $A$, in the parametric function that defines $f(x)$ 
	needs to be set to half of the magnitude obtained using this paradigm in 
	order for the curve given by that function to have the magnitude that it is 
	supposed to have.
	\subsection[Relationships]{Relationships}
	Let $u(x)$ be defined as the unit step function and let $f(x):=\sin x$
	The average value given by $f(x)$, denoted as 'AVG', from $0$ to $\pi$ is 
	given by:
	$$\text{AVG} = \frac{\int_{0}^{\pi}f(x)\cdot dx}{\pi} = \frac{-\cos \pi + 
	\cos 0}{\pi} = \frac{1 + 1}{\pi} = 0.637 $$
	The mean square (MS) part of the root mean square (RMS) is given by:
	$$ MS = \frac{\int_{0}^{\pi}\sin^2x\cdot dx}{\pi}$$
	And the RMS is equal to the square root of MS.
	$$ RMS = \sqrt{MS} = 0.707$$
	\paragraph[Relation3]{Now let the function,} $f(x)$ be redefined as 
	$f(x):=f_1(x)+f_2(x)$, and let $f_1(x)\coloneqq\big(u(x)-u(x-a)\big)x$ and
	$f_2(x):=-f_1(x-a) + a$.
	$$ \text{AVG}=\frac{\int_{0}^{2a}f(x)\cdot dx}{2a} = 
	\frac{\int_{0}^{a}f_1(x)\cdot dx + \int_{a}^{2a}f_2(x) \cdot dx}{2a}$$
	$$ \text{AVG} = \frac{0.5a^2 + \int_{a}^{2a}(2a-x) \cdot dx}{2a}$$
	$$ \text{AVG} = \frac{0.5a^2 + 4a^2 - 2a^2 - 2a^2 + 0.5a^2}{2a}$$
	$$ \text{AVG} = \frac{a^2}{2a} = 0.5a$$
	The notion that the curve given by $f_1(x)$ is equal to the curve given by 
	$f_2(x)$ reflected about the vertical line given by $x=a$ implies that:
	$$ \int_{0}^{a}f_1(x) \cdot dx = \int_{a}^{2a}f_2(x)\cdot dx$$
	This means that another formula to obtain the average is:
	$$ \text{AVG} = \frac{2\int_{0}^{a}f_1(x) \cdot dx}{2a} = 
	\frac{\int_{0}^{a}f_1(x) \cdot dx}{a}$$
	$$ \text{AVG} = \frac{0.5a^2}{a} = 0.5a$$
	To find the $MS$, first let $g(x):=g_1(x)+g_2(x)$, and let 
	$g_1(x):=\big(u(x)-u(x-a)\big)x^2$ and 
	$g_2(x):=\big(u(x-a) - u(x-2a)\big)(x-2a)^2$. \\ \\
	The notion that the curve given by $g_1(x)$ is equal to the curve given by 
	$g_2(x)$ reflected about the vertical line given by $x=a$ implies that:
	$$ \int_{0}^{a}g_1(x) \cdot dx = \int_{a}^{2a}g_2(x)\cdot dx$$
	This means that:
	\begin{align*}
		\text{MS}&=\frac{\int_{0}^{a}g_1(x)\cdot dx}{a} \\
		 &= \frac{a^3}{3a} = \frac{a^2}{3}
	\end{align*}
	Finally,
	$$ \text{RMS} = \sqrt{\text{MS}} = \frac{a}{\sqrt{3}} = 0.577a$$
\end{document}
