\documentclass{article}
\usepackage{amsmath}
\usepackage{amssymb}
\begin{document}
	\section[Series-Parallel]{Resonance in Series-Parallel Circuits}
	Recall that the resonant frequency is given by:
	\begin{equation}\label{eq:res-freq}
		f_{\text{res}}=\frac{1}{2\pi \sqrt{LC}}
	\end{equation}
	The impedance of a parallel circuit with resonant frequency is equal to 
	infinity (assuming there is zero resistance, which only happens in a super 
	conductor), which implies that the current is equal to zero.  Conversely, 
	the impedance is equal to zero -- again, assuming zero resistance -- under 
	the same conditions, implying that the current is of infinite magnitude or 
	amplitude.
	\subsection[High-Res]{Calculating the Resonant Frequency of a 
	High-Resistance Circuit}
	Suppose there is a circuit with an AC power supply of 
	$e:=E_T=1\text{V}\angle0$, a capacitor with a capacitance of 
	$C=10\mu\text{F}=10^{-5}\text{F}$ in parallel with a resistor with a 
	resistance of $R=100\Omega$ and an inductor with an inductance of 
	$L=100\text{mH}=0.1\text{H}$.  This supposed circuit is of the form 
	$C//(R--L)$.\\ \\
	The reactance of each component is given by:
	\begin{align*}
		X_C &= \frac{1}{2\pi f C} \\
		X_R &= R \\
		X_L &= 2\pi f L
	\end{align*}
	And the impedance:
	\begin{align*}
		Z_C &= X_C \angle-90^{\circ} \\
		Z_R &= X_R \angle 0 \\
		Z_L &= X_L \angle 90^{\circ}
	\end{align*}
	The total impedance is given by:
	\begin{equation}\label{eq:total-imp}
		Z_T = \frac{1}{\frac{1}{Z_C}+\frac{1}{Z_R+Z_L}}
	\end{equation}
	Via Ohm's Law,
	$$ E_T = Z_TI$$
	If we want to find the total current, denoted as '$I$', as a function of 
	the frequency, denoted as '$f$', then we need to multiply throughout by the 
	denominator of $Z_T$ in equation (\ref{eq:total-imp}).
	$$ I = E_T\left(\frac{1}{Z_C} + \frac{1}{Z_R+Z_L}\right)$$
	Let $Z_{R--L}:=Z_R + Z_L=R + j \cdot 2\pi f L$.  In polar form:
	$$ Z_{R--L} = \sqrt{R^2 + 4(\pi f L)^2}\angle \arctan \frac{2 \pi f L}{R}$$
	The reciprocal of the above:
	$$ \frac{1}{Z_{R--L}}=\frac{1}{\sqrt{R^2 + 4(\pi f L)^2}}\angle-\arctan 
	\frac{2 \pi f L}{R}$$
	In rectangular form:
	\begin{align*}
		\frac{1}{Z_{R--L}} &= 
		\frac{1}{|Z_{R--L}|}\left(\frac{R}{|Z_{R--L}|} - j\cdot\frac{2\pi f 
		L}{|Z_{R--L}|}\right) \\
		 &= \frac{R-j\cdot 2\pi f L}{R^2 + 4(\pi f L)^2}
	\end{align*}
	And, for the other term of the denominator of $Z_T$:
	$$ \frac{1}{Z_C} = 2\pi f C \angle 90^{\circ}$$
	In rectangular form:
	$$ \frac{1}{Z_C} = j \cdot 2\pi f C$$
	So the denominator is given by:
	\begin{align*}
		\frac{1}{Z_C} + \frac{1}{Z_R + Z_L} &= \frac{R}{R^2+4(\pi f L)^2} + j 
		\cdot 2 \pi f\left(C - \frac{L}{R^2+4(\pi f L)^2}  \right) \\
		 &= \frac{R + j \cdot 2\pi f \Big(C\big(R^2+4(\pi f L)^2\big) - 
		 L\Big)}{R^2+4(\pi f L)^2}
	\end{align*}
	And so the magnitude of the current as a function of the frequency is given 
	by:
	$$ |I|(f) = E_T \frac{\sqrt{R^2 + 4(\pi f)^2\Big(C\big(R^2+4(\pi f 
	L)^2\big) - L\Big)^2}}{R^2+4(\pi f L)^2}$$
	Now let '$V_\theta$' and '$W_\theta$' be defined as:
	$$ V_\theta:=V_\theta(f):=2\pi f$$
	$$ W_\theta:=W_\theta(f):=V_\theta^2$$
	Also note that the reactances of the capacitor and the inductor are a 
	function of the frequency.
	$$ X_C = X_C(f) = \frac{1}{V_\theta (f)C}$$
	$$ X_L = X_L(f) = V_\theta (f)L$$ 
	It then follows that $|I|(f)$ can be more concisely written as:
	$$ |I|(f) = E_T \frac{\sqrt{R^2 + W_\theta\big(C(R^2 +X_L^2 ) - 
	L\big)^2}}{R^2 + X_L^2}$$
	The amplitude at resonant frequency is:
	$$ |I|(f_{\text{res}}=159.159 \text{Hz}) \approx 7.071 \text{mA}$$
	To find the maximum amplitude of $I$ and the frequency at which it occurs, 
	we may need to find the root(s) of the derivative of $|I|(f)$.
	Let '$N$' and '$D$' be defined as:
	$$ N:=\sqrt{R^2 + 4(\pi f)^2\Big(C\big(R^2+4(\pi f 
		L)^2\big) - L\Big)^2}$$
	$$ D:=R^2+4(\pi f L)^2$$
	\begin{align*}
		|I|'(f) =&\frac {4\pi^2 E_T f}{N\cdot D} \Bigg(\Big(C\big(R^2+4(\pi f 
		L)^2\big) - L\Big)^2 +  2C(2\pi Lf)^2\Big(C\big(R^2+4(\pi f 
		L)^2\big) - L\Big)\Bigg) \\
		 -& \hspace{1ex}E_T \frac{8\pi^2L^2 f N}{D^2}
	\end{align*}
	Before even finishing the job of completely finding the derivative, it 
	appears that finding the roots would require finding the roots of a degree 
	six polynomial.  This will require some numerical method.
	\paragraph[Swap]{Now suppose the resistor} is on the other branch of the 
	parallel circuit.  Our supposed circuit now has the form: $(C--R)//L$.  The 
	amplitude of the current as a function of the 
	frequency is given by:
	$$ |I|(f) = E_T\left| \frac{1}{Z_C+Z_R} + \frac{1}{Z_L} \right|$$
	$$ \frac{1}{Z_L} = \frac{1}{2\pi f L}\angle-90^{\circ} = - j\cdot 
	\frac{1}{2\pi f L}$$
	$$ \frac{1}{Z_C + Z_R} = \frac{1}{X_R - j \cdot X_C} =  
	\frac{1}{\sqrt{X_R^2 + X_C^2}}\angle-\arctan -\frac{X_C}{X_R}$$
	Let '$Z_1$' be defined as: $Z_1:=Z_C+Z_R$.
	$$ \frac{1}{Z_1} = \frac{1}{|Z_1|} \left( \frac{X_R}{|Z_1|} + j 
	\frac{X_C}{|Z_1|}\right) = \frac{X_R + j \cdot X_C}{X_R^2 + X_C^2}$$
	$$ \frac{1}{Z_1} + \frac{1}{Z_L} = \frac{X_R}{|Z_1|^2} + j \left( 
	\frac{X_C}{|Z_1|^2} - \frac{1}{X_L}\right)$$
	$$ \left| \frac{1}{Z_1} + \frac{1}{Z_L} \right| = \sqrt{\frac{R^2}{|Z_1|^4} 
	+ \left(\frac{X_C}{|Z_1|^2} - \frac{1}{X_L}\right)^2}$$
	\subsection[LC Series]{Series LC Circuits}
	Suppose we have a circuit of the form: $\text{R1}$--C--L//$\text{R2}$.  
	The values of $C$ 
	and $L$ are the same as the previous subsection, while $R_1=1\Omega$ and 
	$R_2=100\Omega$.  To find the amplitude of the current as a function of the 
	frequency we first need to look at the total impedance of our circuit.
	\begin{equation}\label{eq:total-imp_2}
		Z_T = Z_{R1} + Z_C + \frac{1}{\frac{1}{Z_L} + \frac{1}{Z_{R2}}}
	\end{equation}
	To make the above equation more concise, let N0:=R1, N1:=C, N2:=L//R2, 
	N2A:=L, and N2B:=R2.  The impedance of each component follows the same 
	labeling/denoting scheme.  I.e. $Z0=Z_{R1}$ denotes the impedance of N0 or 
	R1 and $Z2A=Z_L$ denotes the impedance of N2A or L.  Thus we have a more 
	concise equation:
	\begin{equation}\tag{\ref{eq:total-imp_2}}
		Z_T = Z0 + Z1 + Z2
	\end{equation} 
	Note that '$Z2$' denotes the impedance of the parallel combination that is 
	in series with the rest of the circuit.
	$$ Z2 = \frac{1}{\frac{1}{Z2A} + \frac{1}{Z2B}}$$
	Recall from the previous subsection that:
	$$ \frac{1}{Z_L} = \frac{1}{V_\theta L}\angle-90^{\circ} = - j \cdot 
	\frac{1}{V_\theta L} = - j \cdot \frac{1}{X_L}$$
	$$ \frac{1}{Z_{R2}} = \frac{1}{R_2}\angle0 = \frac{1}{R_2}$$
	Adding them gives:
	$$ \frac{1}{Z2A} + \frac{1}{Z2B} = \frac{1}{R_2} - j \cdot \frac{1}{X_L}$$
	Let $D_{Z2}$ be defined as:
	$$ D_{Z2} := \frac{1}{Z2A} + \frac{1}{Z2B}= \frac{1}{R_2} - j \cdot 
	\frac{1}{X_L}$$
	The polar form is given by:
	$$ D_{Z2} = |D_{Z2}|\angle\arctan - \frac{R_2}{X_L}$$
	$$ |D_{Z2}| = \sqrt{\frac{1}{R_2^2} + \frac{1}{X_L^2}}$$
	It then follows that:
	$$ Z2 = \frac{1}{D_{Z2}} = \frac{1}{|D_{Z2}|}\angle-\arctan - 
	\frac{R_2}{X_L}$$
	To add the above impedance value with the impedance values of the 
	components that it's corresponding component is in series with, we need to 
	convert it into rectangular form.
	$$ Z2 = \frac{1}{|D_{Z2}|}\left( \frac{1}{R_2 |D_{Z2}|} + j \cdot 
	\frac{1}{X_L |D_{Z2}|} \right)$$
	And of course, $Z1$ in rectangular form is:
	$$ Z1 = - j \cdot X_C$$
	$$ Z_T = R_1 + \frac{1}{R_2|D_{Z2}|^2} + j \cdot \left( 
	\frac{1}{X_L|D_{Z2}|^2} - X_C \right)$$
	The magnitude of the total impedance is given by:
	\begin{equation}\label{eq:mag-total-imp_2}
		|Z_T| = \sqrt{\left( R_1 + \frac{1}{R_2|D_{Z2}|^2} \right)^2 + \left( 
		\frac{1}{X_L|D_{Z2}|^2} - X_C \right)^2}
	\end{equation}
	The amplitude of the current as a function of frequency is thus given by:
	$$ |I|(f) = \frac{E_T}{|Z_T|}$$
	\paragraph[Swap]{Now suppose } the inductor and capacitor swap places.  Our 
	supposed circuit now has the form of: R1--C//R2--L=R1--L--C//R2.  Note that 
	the category of circuit components, (CC, --, //), is commutative, hence the 
	equality of the two circuit expressions, both of which describe the 
	supposed circuit.  \\ \\
	The total impedance is:
	$$ Z_T=Z0+Z1+Z2$$
	$$ Z0=R_1\angle0=R_1$$
	$$ Z1=Z_L=X_L\angle90^{\circ}=j \cdot X_L$$
	Let $D_{Z2}$ be defined as:
	$$ D_{Z2} := \frac{1}{Z_{R2}} + \frac{1}{Z_C} = \frac{1}{R_2}\angle0 + 
	V_\theta C \angle 90^{\circ}$$
	And in rectangular form:
	$$ D_{Z2}=\frac{1}{R_2} + j \cdot V_\theta C$$
	In polar form:
	$$ D_{Z2}=|D_{Z2}|\angle\arctan (V_\theta C \cdot R_2)$$
	$$ |D_{Z2}| = \sqrt{\frac{1}{R_2^2} + W_\theta C^2}$$
	Finally, for our third term:
	$$ Z2 = \frac{1}{D_{Z2}}= \frac{1}{|D_{Z2}|}\angle - \arctan (V_\theta C 
	\cdot R_2)$$
	In rectangular form:
	$$ Z2 = \frac{1}{|D_{Z2}|}\left( \frac{1}{|D_{Z2}| \cdot R_2} - j \cdot 
	\frac{V_\theta C}{|D_{Z2}|} \right)$$
	The total impedance in rectangular form is:
	$$ Z_T =  R_1 + \frac{1}{|D_{Z2}|^2 \cdot R_2} + j \cdot \left( X_L - 
	\frac{V_\theta C}{|D_{Z2}|^2} \right)$$
	\begin{equation}\label{eq:mag-total-imp_3}
		|Z_T| = \sqrt{\left( R_1 + \frac{1}{|D_{Z2}|^2 \cdot R_2} \right)^2 + 
		\left( X_L - \frac{V_\theta C}{|D_{Z2}|^2} \right)^2}
	\end{equation}
	Notice how the positions of $X_L$ and $X_C$ our swapped in the above 
	equation with respect to their previous positions in equation 
	(\ref{eq:mag-total-imp_2}), which gives the magnitude of the total 
	impedance before the corresponding components swapped places in the circuit 
	layout.\\ \\
	And, finally,
	$$ |I|(f) = \frac{E_T}{|Z_T|}$$
\end{document}
