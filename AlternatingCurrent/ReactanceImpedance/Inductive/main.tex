\documentclass{article}
\usepackage{amsmath}
\usepackage{amssymb}
\begin{document}
	\section[Induction]{AC Inductor Circuits}
	\subsection[Equation]{Relationship between Inductance and Voltage}
	\begin{equation}\label{eq:inductor}
		e(t)=Li'(t)=L\frac{di}{dt}
	\end{equation}
	Here, '$L$' denotes the inductance in Henry's(H), and '$i$' and '$e$' 
	denote the current and voltage in amps(A) and volts(V) respectively.
	\subsection[Reactance]{Reactance vs. Resistance}
	The reactance, denoted as '$X_L$, of an inductor is measured in Ohms -- it 
	corresponds to the resistance of a resistor -- and is given by the equation:
	\begin{equation}\label{eq:reactance}
		X_L=V_\theta L = 2\pi f L
	\end{equation}
	In other words, the reactance of an inductor with an inductance of $L$ is 
	equal to the angular velocity ($V_\theta$) multiplied by the inductance.  
	The angular velocity is equal to the frequency in Hertz, denoted as '$f$', 
	multiplied by 
	$2\pi$ radians or $360$ degrees.
	\paragraph[Proof]{To prove this,} first let $e(t):=|E|\cdot\sin V_\theta 
	t$, where '$E$' denotes the complex number representation of the AC 
	waveform given by $e(t)$, and subsequently, '$|E|$' denotes the magnitude 
	or amplitude of said waveform. \\ \\
	Via equation (\ref{eq:inductor}) and this definition:
	$$ i = \frac{1}{L}\int e(t) \cdot dt = - \frac{|E|}{LV_\theta} \cos 
	V_\theta t$$
	$$ iLV_\theta = - |E|\cdot\cos V_\theta t = |E|\cdot \sin \big((V_\theta 
	(t-0.5\pi)\big) $$
	$$ V_\theta L = X_L = \frac{|E|\angle-90^{\circ}}{i}$$
	Since the current is given by:
	$$ i = \frac{|E|}{X_L}\angle-90^{\circ}$$
	And, via a further generalization of Ohm's Law:
	$$ \frac{e}{i} = Z_L$$
	Here, '$Z_L$' denotes the impedance of the inductor which is a complex 
	number with the reactance being equal to the magnitude of.
	$$ \frac{V_\theta L|E|\angle0}{|E|\angle-90^{\circ}} = Z_L = V_\theta L 
	\angle 90^{\circ} = X_L\angle90^{\circ}$$
	\section[In Series]{Series Resistor-Inductor Circuits}
	Suppose there is a circuit with an AC power supply of 
	$e:=E_T=10\text{V}\angle0$ with a frequency of $f=60 \text{Hz}$, and a 
	resistor with a resistance of $R=5\Omega$ in series with a inductor with an 
	inductance of $L=10\text{mH}=0.01\text{H}$. \\ \\
	Via KVL,
	\begin{equation}\label{eq:series-circuit}
		e=E_T=E_R+E_L
	\end{equation}
	There are two ways to solve for the current, denoted as '$i$'.  One 
	involves representing the AC waveforms associated with the quantities of 
	the circuit and circuit components as complex numbers.  The other involves 
	solving the differential equation derived from the circuit equation derived 
	from KVL.
	\subsection[Complex]{Using the Complex Number Method}
	Via Ohm's Law and equation (\ref{eq:series-circuit}),
	\begin{equation}\tag{\ref{eq:series-circuit}}
		e=Ri + X_Li=i(R+X_L)
	\end{equation}
	Which leaves us with:
	$$ i=\frac{e}{R+X_L}$$
	The impedance of the inductor is given by:
	\begin{align*}
		X_L &= 
		120\pi\frac{\text{rad.}}{\text{s}}\cdot0.01\text{H}\angle90^{\circ} \\
		 &= 1.2\pi \Omega \angle90^{\circ}
	\end{align*}
	In rectangular form:
	$$ X_L = j \cdot 1.2\pi \Omega$$
	The total impedance in rectangular form is given by:
	$$ X = R + X_L = 5\Omega + j \cdot 1.2\pi \Omega$$
	And in polar form:
	$$ X = |X|\angle\theta$$
	where,
	$$ |X| = 6.26\Omega$$
	$$ \theta = \arcsin \frac{1.2\pi}{|X|} = 37.03^{\circ}$$
	Which means the current is given by:
	$$ i = \frac{10\text{V}\angle0}{6.26\Omega\angle37.03^{\circ}} = 
	1.597\text{A}\angle-37.03^{\circ}$$
	Finally, as a trig. function:
	$$i:=i(t)=1.597\text{A}\cdot \sin \big(120\pi(t-0.205\pi)\big)$$
	\subsection[DiffEq.]{Using the Differential Equation Method}
	The circuit equation can be rewritten as a differential equation with the 
	current being the dependent variable and time being the independent 
	variable.
	\begin{equation}\tag{\ref{eq:series-circuit}}
		10\text{V}\sin 120\pi t = 0.01\text{H}i'(t) + 5\Omega i(t)
	\end{equation}
	The solution will be of the following form:
	$$ i(t) = a\cdot\sin V_\theta t + b\cdot\cos V_\theta t + Ae^{mt}$$
	Here, $m$ is the root to the auxiliary equation, while $A \in \mathbb{R}$ 
	is an arbitrary constant to be solved for using an initial value problem 
	(IVP).
	$$ 0.01m + 5 = 0$$
	\begin{equation}\label{eq:m}
		m = -\frac{5\Omega}{0.01\text{H}} = -500 \frac{\Omega}{\text{H}}
	\end{equation}
	To solve for '$a$' and '$b$':
	$$ a5\cdot\sin V_\theta t - bV_\theta0.01\sin V_\theta t = 
	10\sin V_\theta t$$
	\begin{equation}\label{eq:trig_1}
		5 a - 0.01 V_\theta b = 10
	\end{equation}
	\begin{equation}\label{eq:trig_2}
		5 b + 0.01 V_\theta a = 0
	\end{equation}
	Via equation (\ref{eq:trig_1}),
	$$ 5a = 10 + 0.01V_\theta b$$
	\begin{equation}\label{eq:a-in-b}
		a = 2 + 0.24\pi b
	\end{equation}
	Via the above and equation (\ref{eq:trig_2}),
	$$ 5b + 1.2\pi (2+0.24\pi b) = 0$$
	$$ 5b + 0.288\pi^2b = - 2.4\pi$$
	\begin{equation}\label{eq:b}
		b = -0.961
	\end{equation}
	And via equations (\ref{eq:b}), (\ref{eq:a-in-b}), and substitution,
	\begin{equation}\label{eq:a}
		a = 1.275
	\end{equation}
	The two terms of the solution containing trig functions as factors can be 
	added up to one term with one trig function.  To do this, first we shall 
	convert the aforementioned terms into their complex number representation 
	in polar form.  Let '$\mathbf{a}$' and '$\mathbf{b}$' be defined as:
	$$ \mathbf{a} := a\angle0$$
	$$ \mathbf{b} := b\angle90^{\circ}$$
	Now we convert them to rectangular form:
	$$ \mathbf{a} = a\cos 0 + j\cdot a\sin0 = a$$
	$$ \mathbf{b} = b(\cos0.5\pi + j \cdot \sin 0.5\pi) = j \cdot b $$
	Adding them gives:
	$$ \mathbf{c}:=\mathbf{a} + \mathbf{b} = a + j\cdot b$$
	And, in polar form:
	$$ \mathbf{c} = c\angle c_\theta$$
	$$ c = \sqrt{a^2 + b^2} = 1.597$$
	$$ c_\theta = \arcsin \frac{b}{c} = -37^{\circ}$$
	Which leaves us with the general solution of:
	\begin{equation}\label{eq:general-i}
		i = a \cdot \sin V_\theta t + b \cdot \cos V_\theta t + Ae^{mt} = 1.597 
		\sin \big(V_\theta (t - 37^{\circ}) \big) + A e^{mt}
	\end{equation}
	Assuming an IVP of $i(0)=0$ leaves us with the specific solution.
	$$ i(0) = c\cdot\sin(-V_\theta37^{\circ}) + A = 0$$
\end{document}
