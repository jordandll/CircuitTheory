\documentclass{article}
\usepackage{amsmath}
\usepackage{amssymb}
\begin{document}
	\section[Capacitor]{AC Capacitor Circuits}
	\subsection[CapvRes]{Capacitors Vs. Resistors}
	
	Capacitor do not behave the same as resistors. Whereas resistors allow 
		a 
	flow of electrons through them directly proportional to the voltage drop, 
	capacitors oppose changes in voltage by drawing or supplying current as 
	they charge or discharge to the new voltage level. \\ \\
	The flow of electrons “through” a capacitor is directly proportional to the 
	rate of change of voltage across the capacitor. This opposition to voltage 
	change is another form of reactance, but one that is precisely opposite to 
	the kind exhibited by inductors.
	\subsection[Equation]{Capacitor Circuit Characteristics}
	Expressed mathematically, the relationship between the current “through” 
	the capacitor and rate of voltage change across the capacitor is as such:
	\begin{equation}\label{eq:capacitor}
		i(t) = C e'(t) = C\frac{de}{dt}
	\end{equation}
	Now let $e(t):=|E|\sin V_\theta t = |E|\angle0$.  It then follows from this 
	definition in conjunction with equation (\ref{eq:capacitor}),
	$$ i(t) = C|E|V_\theta \cdot \cos V_\theta t = C|E|V_\theta \cdot \sin 
	\big(V_\theta(t + 0.5\pi)\big)$$
	$$ i(t) = CV_\theta|E|\angle90^{\circ}$$
	The above equation in conjunction with the definition of $e(t)$ implies 
	that the current leads the voltage by $90$ degrees, or, to put it another 
	way, the voltage lags the current by $90$ degrees.
	\subsection[Reactance]{A Capacitor's Reactance}
	The reactance of a capacitor with a capacitance of $C$ farads is given by:
	\begin{equation}\label{eq:reactance}
		X_C=\frac{1}{V_\theta C} = \frac{1}{2\pi f C}
	\end{equation}
	Via Ohm's Law:
	$$ e = X_C i$$
	$$ X_C = \frac{e}{i} = \frac{|E|\angle0}{CV_\theta|E|\angle90^{\circ}}$$
	$$ X_C = \frac{1}{CV_\theta} \angle-90^{\circ}$$
	\section[In Series]{Series Resistor-Capacitor Circuits}
	In the last section, we learned what would happen in simple resistor-only 
	and capacitor-only AC circuits. Now we will combine the two components 
	together in series form and investigate the effects. \\ \\
	Suppose we have a circuit with a resistor with a resistance of $R=5\Omega$ 
	in series with a capacitor with a capacitance of $C=100\mu\text{F}$ and an 
	AC power supply of $e:=E_T=10\text{V}\angle0$ with a frequency of $f=60 
	\text{Hz}$. \\ \\
	Via KVL,
	$$ E_T = e = E_R + E_C$$
	Via the above equation in conjunction with Ohm's Law,
	$$ 10\text{V}\angle0 = Ri + X_Ci = i(R + X_C)$$
	$$ i = \frac{10\text{V}\angle0}{R + X_C}$$
	\subsection[Impedance]{Impedance Calculations}
	The impedance of the capacitor in polar form is given by:
	$$ X_C = \frac{s}{100\mu\text{F}\cdot 120 \pi} \angle -90^{\circ} = 
	26.52\Omega\angle-90^{\circ}$$
	And in rectangular form:
	$$ X_C = j\cdot 26.52\Omega\sin-0.5\pi = - j \cdot 26.52 \Omega$$
	The total impedance in rectangular form is given by:
	$$ X = R + X_C = 5\Omega - j \cdot 26.52 \Omega$$
	And, in polar form:
	$$ X = |X|\angle\theta$$
	$$ |E| = 26.99 \Omega$$
	$$ \theta = \arcsin -\frac{26.52\Omega}{|E|} = -79.29^{\circ}$$
	The current is thus:
	$$ i = 
	\frac{10\text{V}\angle0}{26.99\Omega\angle-79.29^{\circ}}=333.44\text{mA}\angle79.29^{\circ}$$
\end{document}
