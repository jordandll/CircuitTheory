\documentclass{article}
\usepackage{amsmath}
\usepackage{amssymb}
\begin{document}
	\section[Arithmetic]{Complex Number Arithmetic}
	The relationship between rectangular and polar form is as follows:
	$$\mathbf{A}= |\mathbf{A}|\angle\theta = A_x + A_yj$$
	Here, $j:=\sqrt{-1}$, denotes the imaginary number, while '$A_x$' and 
	'$A_y$' denote the $x$ and $y$ complements of $\mathbf{A}$ as a vector and 
	the real and imaginary components of $\mathbf{A}$ as a complex number.  
	They are given by:
	$$ A_x = |\mathbf{A}|\cos\theta$$
	$$ A_y = |\mathbf{A}|\sin\theta$$
	Which leaves us with the rectangular form of:
	$$ \mathbf{A} = |\mathbf{A}|(\cos\theta + j\cdot\sin\theta)$$
	\subsection[Addition]{Addition and Subtraction}
	It is best to do this in rectangular form.
	$$\mathbf{A} + \mathbf{B} = (A_x + B_x) + (A_y+B_y)j$$
	\subsection[Multiplication]{Multiplication and Division}
	It is best to do this in polar form.
	$$\mathbf{A}\mathbf{B}=(|\mathbf{A}|\angle\varphi)(|\mathbf{B}|\angle\lambda)=|\mathbf{A}|
	|\mathbf{B}|\angle(\varphi+\lambda)$$
	It makes sense that the magnitude is equal to the product of the two 
	magnitudes of the complex numbers being multiplied, but what about the 
	angle being the sum of the two numbers' angles?
	$$ 
	\mathbf{A}\mathbf{B}=|\mathbf{A}||\mathbf{B}|(\cos\varphi+j\cdot\sin\varphi)(\cos\lambda
	 + 
	j\cdot\sin\lambda)$$
	$$\mathbf{A}\mathbf{B}= (A_x+A_yj)(B_x+B_yj) = A_xB_x + j(B_xA_y + A_xB_y) 
	- A_yB_y$$
	$$ AB_x = A_xB_x - A_yB_y = |\mathbf{AB}|\cos(\varphi + \lambda) = 
	|\mathbf{A}||\mathbf{B}|\cos(\varphi + \lambda)$$
	The above is consistent with the dot product rule in conjunction with the 
	notion that if either $\mathbf{A}$ or $\mathbf{B}$ were to be reflected 
	about the x-axis, the resulting vector would have the same x-complement 
	while the y-complement would be equal to the additive inverse of it's 
	previous value, the one before this reflection event. \\ \\
	Let $\mathbf{B'}$ be defined as:
	$$\mathbf{B'}:=|\mathbf{B}|\angle(-\lambda)=B_x-B_yj$$
	$\mathbf{B'}$ is equal to $\mathbf{B}$ reflected about the x-axis, 
	therefore the angle between $\mathbf{B'}$ and $\mathbf{A}$ is equal to 
	$\varphi+\lambda$.  Via the dot product formula:
	$$ \mathbf{A\cdot B'} = A_xB'_x + A_yB'_y = A_xB_x-A_yB_y = 
	|\mathbf{A}||\mathbf{B'}|\cos(\varphi+\lambda)$$
	The aforementioned reflection event of course does not change the 
	magnitude, therefore,
	$$ |\mathbf{A}||\mathbf{B'}|\cos(\varphi+\lambda) = 
	|\mathbf{A}||\mathbf{B}|\cos(\varphi+\lambda)$$
	And now, for the y-complement (imaginary component):
	$$ AB_y = B_xA_y + A_xB_y = |\mathbf{AB}|\sin(\lambda+\varphi) = 
	|\mathbf{A}||\mathbf{B}|\cos(0.5\pi - \lambda - \varphi)$$
	Let $\mathbf{A'}$ be defined as:
	$$ \mathbf{A'} := |\mathbf{A}|\angle(0.5\pi - \varphi)$$
	It then follows that:
	$$ \mathbf{A'} = A_y + A_xj$$
	Note the duality between $\mathbf{A}$ and $\mathbf{A'}$ with respect to the 
	real and imaginary components of each;  The real component of $\mathbf{A'}$ 
	is equal to the imaginary component of $\mathbf{A}$ and vice versa. \\ \\
	The angle between $\mathbf{A}$ and $\mathbf{B}$, denoted as 
	'$\theta(\mathbf{A},\mathbf{B})$', is equal to $|\varphi-\lambda|$.  
	Similarly, the angle between $\mathbf{A'}$ and $\mathbf{B}$ is equal to 
	$|0.5\pi-\varphi-\lambda|$.
	$$ \theta:=\theta(\mathbf{A'},\mathbf{B})=|0.5\pi-\varphi-\lambda|$$
	Via the dot product formula,
	\begin{align*}
		\mathbf{A'\cdot B} &= A'_xB_x + 
		A'_yB_y=|\mathbf{A'}||\mathbf{B}|\cos\theta \\
		 &= A_yB_x + A_xB_y = |\mathbf{A}||\mathbf{B}|\cos\theta
	\end{align*}
	The above is consistent with the equation that gives $AB_y$.
	\section[Examples]{Some Examples Circuits with AC}
	Find the sum of the following voltages:
	$$ 15\text{V}\angle0 + 12\text{V}\angle35^{\circ} + 
	22\text{V}\angle-64^{\circ} = E_T$$
	$$ E_T = 15\text{V}\angle0 + 12\text{V}\angle0.194\pi + 
	22\text{V}\angle-0.356\pi$$
	Each voltage source can be expressed in rectangular form.
	$$ E_1 := 15\text{V}\angle0=15\text{V}(\cos0 + j\cdot\sin 0) = 15\text{V}$$
	$$ E_2 := 12\text{V}\angle0.194\pi = 12\text{V}(\cos 0.194\pi + j\cdot\sin 
	0.194\pi)$$
	$$ E_3:=22\text{V}\angle-0.356\pi = 22\text{V}(\cos-0.356\pi + 
	j\cdot\sin-0.356\pi)$$
	$$ E_T = 34.46\text{V} - j \cdot 12.92 \text{V} $$
	And now, back to polar form:
	$$ |E_T| = \text{V}\cdot\sqrt{34.46^2 + 12.92^2} = 36.80 \text{V}$$
	$$ \sin E_\theta = -\frac{12.92}{|E_T|}$$
	$$ E_T = |E_T|\angle E_\theta = 36.80\text{V}\angle-0.114\pi = 
	36.80\text{V}\angle-20.55^{\circ}$$
	
\end{document}
